\documentclass[a4paper,10pt]{article}
\usepackage[utf8]{inputenc}
\usepackage[left=1cm, right=1cm, top =2cm, bottom=2cm]{geometry}
%opening
\title{Arduino}
\author{Salem \textsc{Harrache} \and Elizabeth \textsc{Paz}}
\date{}

\begin{document}
\maketitle
\section{Fonctions}

\subsection{Digital I/O}

\begin{tabular}{|l|l|}
\hline
pinMode(pin, mode) & local ledpin = pio.X \\
 & pio.pin.setdir(ledpin,mode) \\ \hline \hline
Configures the specified pin to behave either as an input or an output & Declared the variable that we are going to use \\
pin - the number of the pin whose mode you wish to set & X- correspond to the pin in that platform \\
mode - either INPUT or OUTPUT & \\ \hline

digitalWrite(pin, value) & pio.pin.sethigh(pin) ou pio.pin.setlow(pin) \\
Write a HIGH or a LOW value to a digital pin  & Set pin to high and low \\ \hline

digitalRead(pin) & pio.port.getval(pin) \\ 
Reads the value from a specified digital pin, either HIGH or LOW &  \\ \hline
\hline \end{tabular}

\subsection{Time}

\begin{tabular}{|l|l|} \hline
delay(ms) & tmr.delay( id, period )\\
Pauses the program for the amount  & Waits for the specified period, then returns \\ 
of time (in ms) specified as parameter & \\
&  id - the timer ID and period - how long to wait (in us)\\
\hline \end{tabular}



\end{document}
