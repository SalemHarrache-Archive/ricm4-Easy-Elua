%%%%%%%%%%%%%%%%%%%%%%%%%%%%%%%%%%%%%%%%%%%%%%%%%%%%%%%%%
\chapter[Exemples]{Exemples} 
\label{exemples}
\section{Blink}

\subsection{Version Arduino}

\begin{table}[h]
\begin{lstlisting}
/*
  Blink
Turns on an LED on for one second, then off for one second, repeatedly.

  This example code is in the public domain.
 */

// Pin 13 has an LED connected on most Arduino boards.
// give it a name:
int led =13;

// the setup routine runs once when you press reset:
void setup(){
    // initialize the digital pin as an output.
    pinMode(led, OUTPUT);
}

// the loop routine runs over and over again forever:
void loop(){
    digitalWrite(led, HIGH);// turn the LED on (HIGH is the voltage level)
    delay(1000);// wait for a second
    digitalWrite(led, LOW);// turn the LED off by making the voltage LOW
    delay(1000);// wait for a second
}
\end{lstlisting}
\caption{Blink: version Arduino}
\end{table}

\subsection{Version Easy-eLua}

\begin{table}[h]
\begin{lstlisting}
--  Blink
--  Turns on an LED on for one second, then off for one second, repeatedly.
require("arduino_wraper")

functionApp:setup()
    self.ledpin= ORANGE_LED -- Pin PD_13 has a LED connected
    pinMode(self.ledpin, OUTPUT)-- Initialize the digital pin as an output.
end

functionApp:loop()
    digitalWrite(self.ledpin, HIGH)-- set the LED on
    delay(1000)-- wait for a second
    digitalWrite(self.ledpin, LOW)-- set the LED off
    delay(1000)-- wait for a second
end

app=App:new("Blink led")
app:run()
\end{lstlisting}
\caption{Blink: version Easy-eLua}
\end{table}


%%%
\newpage
\section{DigitalRead}

\subsection{Version Arduino}

\begin{table}[h]
\begin{lstlisting}
/*
  DigitalReadSerial
 Reads a digital input on pin 2, prints the result to the serial monitor 

 This example code is in the public domain.
 */

// digital pin 2 has a pushbutton attached to it. Give it a name:
intpushButton=2;

// the setup routine runs once when you press reset:
void setup(){
    // initialize serial communication at 9600 bits per second:
    Serial.begin(9600);
    // make the pushbutton's pin an input:
    pinMode(pushButton, INPUT);
}

// the loop routine runs over and over again forever:
void loop(){
    // read the input pin:
    intbuttonState=digitalRead(pushButton);
    // print out the state of the button:
    Serial.println(buttonState);
}
\end{lstlisting}
\caption{DigitalRead: version Arduino}
\end{table}
\newpage
\subsection{Version Easy-eLua}

\begin{table}[h]
\begin{lstlisting}
-- DigitalReadSerial
-- Reads a digital input on pin PA0 (B1 user), prints the result to the serial
-- monitor
require("arduino_wraper")

-- the setup routine runs once when you press reset
functionApp:setup()
    Serial2:begin(9600)-- initialize serial com at 9600 bits per second
    pinMode(USER_BTN, INPUT)-- Initialize the digital pin as an output.
end

functionApp:loop()
    buttonState=digitalRead(USER_BTN)-- read the input pin
    Serial2:print(buttonState)-- print out the state of the button
end

app=App:new("DigitalRead Serial")
app:run()
\end{lstlisting}
\caption{DigitalRead: version Easy-eLua}
\end{table}