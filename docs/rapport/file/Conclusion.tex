%%%%%%%%%%%%%%%%%%%%%%%%%%%%%%%%%%%%%%%%%%%%%%%%%%%%%%%%%
\chapter[Conclusion]{Conclusion - Bilan de l’expérience} \label{chap:conclusion}

Ce projet aura été pour nous un premier contact concret avec le monde de l’embarqué. Malgré un démarrage lent, dû à une grande phase de formation, 
le projet est au final opérationnel et c’est ce qui compte. L’expérience était vraiment enrichissante et a été propice aux nouvelles réflexions sur 
le rôle de l’ingénieur dans la recherche, l'analyse et la conception d’API utilisable par les programmeurs. Pour une fois, on ne se situe non
 pas dans le rôle de l’ingénieur-technicien, mais dans celui de l’ingénieur simplement.

Lors du développement, notre principale règle était de coder proprement et efficacement avec le minimum de lignes de code. 
Plus un code est petit et lisible, plus il est facile à maintenir et à garder cohérent (ne pas avoir de débordement, 
de comportements inattendus, etc\ldots).

Ce projet ne nous a pas seulement appris de nouvelles notions de programmation et conceptions, mais également à 
rechercher efficacement l’information, à prendre contact avec des personnes diverses qui travaillent à l’autre bout du monde, 
à jauger de la viabilité d’un projet et savoir si on l’utilise ou pas dans le nôtre.

Concernant la démarche de travail, on a pu mettre en place une organisation plus flexible par rapport à celle des TPs classiques. 
En effet, si l'autonomie dans un TP est un plus indéniable qui contribue à la réussite, ici elle se trouve au cœur de 
la gestion et du développement du projet. Les différentes phases du projet n'étaient pas figées. L'essentiel était d'atteindre 
l'objectif et par conséquent d’être capable de prendre en  charge les différentes tâches tout en étant le plus productif possible.  

