%%%%%%%%%%%%%%%%%%%%%%%%%%%%%%%%%%%%%%%%%%%%%%%%%%%%%%%%%
\chapter[Fonctions portées]{Fonctions portées}
\label{chap:chap6}


L'environnement eLua est très riche.
La plupart des portages qu’on a réalisés sont assez courts et dépassent rarement les 10 lignes. Voici les principales fonctions portées.

\section{Entrées/Sorties numériques}

Ce sont les fonctions de base d’entrée sortie qui vont permettre d’interagir avec l’extérieur.  Trois fonctions ont été réalisées: 

\begin{description}
 \item[pinMode(): ] Cette fonction spécifie le mode (INPUT ou OUTPUT) d’une broche. 
Cette fonction existe déjà dans eLua, mais elle porte un autre nom, le portage est donc très court:

\begin{table}[h]
\begin{lstlisting}
function pinMode(pin, mode)
           if mode == OUTPUT or mode == INPUT then
               pio.pin.setdir(mode, pin)
           end
end
\end{lstlisting}
\caption{Fonction $pinMode$}
\end{table}

\item[digitalWrite(): ] Envoie la valeur HIGH ou LOW (respectivement 1 ou 0) sur une broche. De même que $pinMode()$ 
le portage ne pose pas de problème puisque cette fonction (comme on peut s’en douter) est disponible dans l’environnement eLua.
\newpage
\begin{table}[h]
\begin{lstlisting}
Function digitalWrite(pin, value)
	    if value == HIGH or value == LOW then
              pio.pin.setval( value, pin )
	    end
end
\end{lstlisting}
\caption{Fonction $digitalWrite$}
\end{table}

\item[digitalRead(): ] DigitalRead permet de lire la valeur de la broche. Cette valeur est soit de 0 soit de 1.
\begin{table}[h]
\begin{lstlisting}
functiondigitalRead(pin)
          return pio.pin.getval( pin )
end
\end{lstlisting}
\caption{Fonction $digitalRead$}
\end{table}


\end{description}

\section{Communication série}

Pour l'entrée/sortie série, nous avons implémenté la classe Serial proposée par Arduino.

\begin{table}[h]
\begin{lstlisting}
-- Serial communication object
SerialPort=Class:new()

function SerialPort:__new(uartid, baud,databits, parity,stopbits)
    -- Initialize SerialPort
    self.uartid = uartid
    self.baud = baud or 115200
    self.databits = databits or 8
    self.parity= parity or uart.PAR_NONE
    self.stopbits=stopbits or uart.STOP_1
end
\end{lstlisting}
\caption{Entrée/Sortie}
\end{table}

La carte STM32F4-DISCOVERY dispose de 6 modules UART (Universal Asynchronous Receiver Transmitter) dont deux synchrones (USART).
 Dans la librairie Arduino, ces modules peuvent être utilisés par des objets nommés \textbf{Serialx} où x est l’identifiant de l’UART. 
Nous disposons donc dans Easy-eLua de 6 objets Serial accessibles directement depuis le programme, puisqu’ils sont initialisés 
directement dans le fichier \textit{arduino\_wrapper.lua}.

\begin{table}[h]
\begin{lstlisting}
-- Define Serials object
Serial0 =SerialPort:new(0)
Serial1 =SerialPort:new(1)
Serial2 =SerialPort:new(2)
Serial3 =SerialPort:new(3)
Serial4 =SerialPort:new(4)
Serial5 =SerialPort:new(5)
\end{lstlisting}
\caption{Objets Serial}
\end{table}

\newpage
La classe dispose bien évidemment d’une méthode $begin()$
qui initialise la connexion série. Dans ce cas les broches ne peuvent plus être utilisées avec $digitalWrite()$ et $digitalRead()$.

\begin{table}[h]
\begin{lstlisting}
function SerialPort:begin(baud)
    -- Setup the serial port
    -- Returns: The actual baud rate set on the serial port.
    -- Depending on the hardware, this might have a different value than
    -- thebaud parameter
    self.baud = baud or self.baud
    return uart.setup(self.uartid,self.baud,self.databits,self.parity,
                                                       self.stopbits)
end
\end{lstlisting}
\caption{Fonction $begin$}
\end{table}

La fonction la plus intéressante (nous ne détaillerons pas les autres ici) c’est bien évidemment la fonction $print()$ 
qui permet d’envoyer des données avec différents formatages possibles:
\newpage
\begin{table}[h!]
\begin{lstlisting}
functionSerialPort:print(value,format)
    -- Prints data to the serial port as human-readable ASCII text.
    -- This method can take many forms. Numbers are printed using 
	an ASCII
    -- character for each digit. Floats are similarly printed 
	as ASCII digits,
    -- defaulting to two decimal places. Bytes are sent as a 
	single character.
    -- Characters and strings are sent as is. For example:
    -- SerialPort.print(78) gives "78"
    -- SerialPort.print(1.23456) gives "1.23"
    -- SerialPort.print('N') gives "N"
    -- SerialPort.print("Hello world.") gives "Hello world."

    -- An optional second parameter specifies the base (format) 
	to use;
    -- permitted values are BIN (binary, or base 2), OCT (octal, 
	or base 8),
    -- DEC (decimal, or base 10), HEX (hexadecimal, or base 16). 
    -- For floating point numbers, this parameter specifies
    -- the number of decimal places  to use. For example:
    -- SerialPort.print(78, BIN) gives "1001110"
    -- SerialPort.print(78, OCT) gives "116"
    -- SerialPort.print(78, DEC) gives "78"
    -- SerialPort.print(78, HEX) gives "4E"
    -- SerialPort.println(1.23456, 0) gives "1"
    -- SerialPort.println(1.23456, 2) gives "1.23"
    -- SerialPort.println(1.23456, 4) gives "1.2346"

    if value == nil then
        return
    end

    if type(value) == "string" then
        uart.write(self.uartid, value)
    end

    if type(value) == "number" then
        if format ~= nil then
            if format == BIN then
                value = numberstring(value,2)
            elseif format == OCT then
                value = numberstring(value,8)
            elseif format == DEC then
                value = numberstring(value,10)
            elseif format == HEX then
                value=string.upper(numberstring(value,16))
            end
        elseif type(format) == "number" and format>=0 then
            value=string.format("%."..format.."f", value)
        else
            -- if value is float
            if value ~= floor(value)then
                value = string.format("%.2f", value)
            else
                value = string.format("%d", value)
            end
        end
        uart.write(self.uartid, value)
    end
end
\end{lstlisting}
\caption{Fonction $print$}
\end{table}
\newpage
Nous sommes restés le plus fidèle possible à la fonction initiale de Arduino.



