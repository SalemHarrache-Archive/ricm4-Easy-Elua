%%%%%%%%%%%%%%%%%%%%%%%%%%%%%%%%%%%%%%%%%%%%%%%%%%%%%%%%%
\chapter[Nouveaux concepts]{Nouveaux concepts}

La structure générale d'un programme Arduino repose sur les méthodes setup() et loop().
Le méthode loop() qui constitue le cœur du programme sera appelée indéfiniment, tandis que la méthode setup() elle, 
sert seulement à initialiser les variables et le contexte d'exécution au début du programme.

Partant de ce constat, on a décidé de donner une orientation objet à nos programmes lua. En effet 
ce choix peut être contestable, mais il nous est paru important pour mettre en place une API efficace.

Lua n'est pas pourvu de type ``class'', mais il permet de programmer le comportement de ses propres objets/types. 
Autrement dit, il est certes très léger, mais c'est un langage tout désigné pour la méta-programmation. 
Le mécanisme de classe, et d’héritage est défini par les lignes suivantes:

\begin{table}[h]
\begin{lstlisting}
function Class:new(super)
    local class,metatable, properties = {},{},{}
    class.metatable=metatable
    class.properties=properties

    function metatable:__index(key)
        local prop = properties[key]
        if prop then
            return prop.get(self)
        elseif class[key] ~=nil then
            return class[key]
        elseif super then
            return super.metatable.__index(self, key)
        else
            return nil
        end
    end

    function metatable:__newindex(key, value)
        local prop = properties[key]
        if prop then
            return prop.set(self, value)
        elseif super then
            return super.metatable.__newindex(self, key, value)
        else
            rawset(self, key, value)
        end
    end

    functionclass:new(...)
        local obj=setmetatable({},self.metatable)
        if obj.__new then
            obj:__new(...)
	end
        return obj
    end
    return class
end
\end{lstlisting}
\caption{Définition du type $Class$}
\end{table} 

\newpage

Grâce à ce nouveau type, nous avons défini la classe série (vu plus haut) mais également la classe centrale: App

\begin{table}[h]
\begin{lstlisting}
-- App
App = Class:new()

function App:__new(name)
    -- Initialize App
    self.name = name
end

function App:setup()
    return
end

function App:loop()
    return
end

function App:run()
    self:println("Run : ".. self.name)
    self:setup()
    while self:condition()do
        self:loop()
    end
    collectgarbage()
end
\end{lstlisting}
\caption{Classe centrale $App$}
\end{table}

Un programme utilisant Easy-eLua n’aura qu'à redéfinir le comportement de setup() et de loop() (voir les exemples dans le chapitre ~\ref{exemples}). 
Ce qu'il faut faire en plus par rapport au modèle Arduino, est l’instanciation d’un nouvel objet $App$, puis on l'appelle à la méthode run(). 
Le lancement est donc explicite. L’idée sous-jacente est la possibilité d’avoir plusieurs applications dans un même programme.
Et le choix de laquelle lancer pourra se faire sur certaines conditions. 

\section{Fonctions relatives au temps}

Deux fonctionnalités intéressantes pour un programme ont été mises au point. 
La première est la durée d’exécution du programme. Depuis combien de temps il tourne. Ça permet généralement de différer dans le temps 
certaines tâches. La seconde est la possibilité d'attendre passivement (delay()), ce qui est vital si on ne veut pas surcharger le CPU 
(attente active avec un while).

Ces fonctionnalités sont plus au moins disponibles dans eLua.
\newpage
\begin{table}[h]
\begin{lstlisting}
function delay(period)
    -- period in milliseconds
    tmr.delay(0,period*1000)
end

function delayMicroseconds(period)
    -- period in us
    tmr.delay(0, period)
end
\end{lstlisting}
\caption{Fonction $delay$}
\end{table}

L'implémentation utilise le module timer de eLua. Lors du lancement de l'application, 
on mémorise le compteur d’un timer, pour que l'on puisse retrouver, par différence, la durée d'exécution:

\begin{table}[h]
\begin{lstlisting}
function App:run()
    self:println("Run : ".. self.name)
    self.start_counter=tmr.start(self.timerid)
    [...]
end
function App:micros()
    -- Number of microseconds since the program started
    time_now=tmr.read(self.timerid)
    return tmr.gettimediff(self.timerid,self.start_counter,time_now)
end
\end{lstlisting}
\caption{Utilisation du module timer}
\end{table}

\section{Le Shell}

Le Shell est un des avantages indéniables de l'environnement lua. On peut l'utiliser pour exécuter un interprète 
lua qui tourne directement sur la carte. On peut ainsi écrire des programmes dynamiquement, et par extension utiliser Easy-eLua par ce moyen.

Voici un exemple ``HelloWorld'' utilisant Easy-eLua: 

\begin{table}[h]
\begin{lstlisting}
eLua#lua
Press CTRL+Z to exitLua
Lua5.1.4  Copyright(C)1994-2011 Lua.org, PUC-Rio
>require("arduino_wraper")
>app=App:new("Hello World!")
>app:run()
Run: Hello World
\end{lstlisting}
\caption{Exemple ``Helllo World''}
\end{table}




