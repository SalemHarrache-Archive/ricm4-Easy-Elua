%%%%%%%%%%%%%%%%%%%%%%%%%%%%%%%%%%%%%%%%%%%%%%%%%%%%%%%%%
\chapter*{Introduction} \label{chap:intro}

Après un quatrième semestre très chargé, l’heure est aux projets innovants. Aujourd'hui vient l'heure de l'expérience et du premier vrai test 
sur nos capacités à concevoir, à développer et à innover.

Parmi ces projets, certains, et le nôtre en particulier, s’adressent avant tout aux développeurs et aux entreprises désireuses de gagner du temps, 
et donc de l’argent. En effet, il est question ici de mettre en place une API, plus particulièrement, de porter l’API Arduino dans un autre environnement
et un autre langage.

Arduino est l’entreprise italienne qui a innové dans le monde de l’embarqué par son matériel simple, peu cher, et open source mais également par sa
 librairie qui fait passer la programmation sur embarqué pour un jeu d’enfants. Cette approche, qui consiste d’une part à offrir une structure et API 
très simple et à proposer l’ensemble en open source a aussi bien séduit les novices que les experts. Nous avons été séduits par cette approche en novices 
que nous sommes.

D'autre part, un autre projet open source dans l’embarqué prend de l’ampleur depuis quelques années: eLua. Ce projet propose un environnement Lua 
pour beaucoup de cartes microélectroniques (MicromintEagle 100, Texas instruments LM3S6965 etc.) qui permet d’écrire des programmes avec le très 
puissant langage Lua.

Notre projet repose sur ces deux projets existant pour faciliter la programmation sur les cartes STM32F4-DISCOVERY de ST-Links. Il consiste en 
le développement et portage d’une partie de l’API Arduino sur l’environnement eLua. Le but souhaité est de permettre à n’importe quel développeur
Arduino de s’y retrouver rapidement dans l’environnement eLua avec toutes les fonctions qu’il connait, déjà utilisables et exploitables.

\newpage
Ce rapport s’articule autour de trois parties. En premier lieu,  nous allons revenir sur l’approche Arduino et l’environnement eLua. 
Ensuite, nous détaillerons le déroulement du projet, la méthode de travail et le travail réalisé. Enfin on terminera par le bilan sur 
l’ensemble du projet.